\section{Related Work}
\label{sec:related-work}

To the best of our knowledge, there is no published work documenting an experiment investigating the effects of physical interaction while sport climbing in \gls{VR}. So far, only \textcite{Tiator2018} have presented a demo for climbing with real props in \gls{VR}. Therefore, the studies we took into account come from two areas of research: effects of (height-induced) anxiety while climbing and anxiety while moving through \glspl{VE}.\pdfmargincomment[avatar=peter,style=korrektur-done]{überarbeitet}

\subsection{Stress and Anxiety in Sport Climbing}
\label{sec:related-work-climbing}

The most recent work is a master thesis looking into fear of falling with sport climbers, and its triggers, namely absolute height above ground, potential fall height, surrounding area and landing area, and finally trust in equipment and belayer \autocites{Bertle2014}[based on][]{Sheel2004}. Concerning potential fall-height, \citeauthor{Bertle2014} argues that more experienced climbers are capable of distinguishing between relevant aspects (fall height and quality of the landing area) and irrelevant (absolute height above ground) factors \autocite{Fryer2013,Draper2012,Sherk2011}. 

\textcites{Bertle2014,Fryer2013,Hardy2007} also examined the effect of climbing styles, namely, \glsing{lead} and \glsing{top-rope}. All found a significantly higher cognitive anxiety in \gls{lead} climbers than in \gls{top-rope} climbers \autocite{Bertle2014,Hardy2007}. Most of the earlier research was focused on the absolute height above ground \autocite{Kunzell2010, Oudejans2009, Pijpers2006, Pijpers2005, Pijpers2003}, however. It follows a summary of studies we used as a reference for the climbing tasks and physiological measures. For more details and additional studies, the interested reader is referred to Table \ref{tab:climbing-summary} in the appendix.

\begin{description}
	\item[\textcite{Pijpers2003, Pijpers2005, Pijpers2006}] A total of three studies, with two experiments each, was conducted by Pipers et al. to measure effects of anxiety induced by fear of falling while climbing. In contrast to \textcite{Hardy2007}, participants did neither climb vertically nor \gls{lead}. Instead, they used a setup where participants only traversed (climb horizontally) an identical route at different heights above ground, while being belayed in \gls{top-rope}. All three studies showed an increase in anxiety-related effects (including effects in behavior) in the high condition compared to the low condition.
	
	\item[\textcite{Hardy2007}] In \citetitle{Hardy2007} they describe three experiments they ran to validate \citeauthor{Eysenck1992}'s Processing Theory, which suggests that stress impairs (quality of performance) less than efficiency ($\frac{\text{effectiveness}}{\text{effort}}$) and therefore increases effort. Their findings suggest that \glsing{lead} causes more anxiety than \glsing{top-rope} and therefore higher effort. Additionally, less anxious climbers transform effort into performing better than highly anxious climbers. The authors assume that self-confident subjects (skillful climbers) are less likely to be impacted by anxiety. Unexpectedly, anxiety before having to climb an unknown route was induced at the same intensity regardless of \glsing{lead} or \glsing{top-rope}.
	
	\item[\textcite{Bertle2014}] The author conducted a study to examine the stress reactions to an impending fall. Conditions considered were climbing style (\glsing{lead}/\gls{top-rope}), and height above ground (low/high). Although the author writes about effects of the potential fall height, it remains unclear how they are determined since he attributes them to climbing style \autocite[35]{Bertle2014}. However, the results show significant cognitive and somatic anxiety reactions in a high (fall) condition compared to a low (fall) condition.
\end{description}

\paragraph{Physiological Measures of Anxiety} In his thesis, \textcite{Bertle2014} summarizes multiple studies with skydivers and base jumpers. Inexperienced skydivers show a raised \gls{HR}, \gls{RR}, and \gls{EDA} throughout a flight which do not decrease until landing \autocite{Woodman2009,Fenz1967}. \textcite{Breivik1998} could not corroborate Eysenck's and Calvo's Arousal Theory \citeyear{Eysenck1992} nor could they find consistent correlations of \gls{HR} to subjective feelings among experienced jumpers as before them \textcite{Fenz1967}. However, they could show that experience (number of jumps) is correlated negatively to arousal \autocite{Breivik1998}, an observation \textcite{Hardy2007} later made again for climbers. \textcite{Roth1996} had found similar results before, that showed rising physiological measures as well as anxiety self-ratings for parachutists in anticipation of a jump---independent of the level of expertise, only at higher levels for novices.

In alignment with these studies, we base our work on the assumption that height above ground in combination with the risk of falling induces anxiety in climbers. Further, we assume that cognitive and the somatic anxiety can be operationalized.

\subsection{Stress and Anxiety in Virtual Environments}

As mentioned earlier, acrophobia is an abnormal fear of heights which in turn is related to the fear of falling \autocite{Bertle2014}. Therefore, research looking into a climbers' fear of falling in \gls{VR} would be of particular interest, but, no such work has been published so far. There are publications, however, regarding acrophobia and its therapy using \gls{VR} (\gls{VRET}). Multiple reviews towards the effectiveness of \gls{VRET} suggest that acrophobia may be treated in vivo as well as in virtuo \autocite{Meyerbroker2010,Powers2008,Krijn2004}.

\textcite{Diemer2016} ran a controlled experiment to compare height reactions of phobics and non-phobics. Both groups (n = 40 each) showed similar physiological arousal when exposed to height \autocite{Diemer2016}. This implicates that we do not need phobics to see effects in our experiment and that we can work with a sample comparable to those of other climbing-related studies listed earlier.

Since we use height above ground as a stressor, the perception of height in \gls{VR} is of particular concern. The extent to which the (simulated) height works as a trigger depends on the effectiveness of the \gls{VE} and thereby the amount of presence it evokes in its users \autocite{Meehan2002}. \textcite{Gandy2010} summarize several studies examining effects of varying levels of detail on presence and perception---with contradicting results. On the one hand, \textcite{Zimmons2003} could not find significant differences in presence or subjective stress when running variations of “the~pit” experiment \autocite{Meehan2001}, ranging from wireframe to radiosity. On the other hand, \textcite{Slater2009} found increased psycho-physiological measures, \gls{HR}, and \gls{SC}, for a higher level of realism (reflections and shadows).

In alignment with \textcite{Gandy2010}, who summarized that a “[virtual] environment does not have to be visually perfect to make users feel present or capable of a motor task”, we recreated a degree of realism comparable to that of “the pit” experiment by \textcite{Meehan2002}. Therefore, we used photo textures, dynamic shadows, and reflections but we kept the amount of details in the 3D models low.\pdfmargincomment[avatar=peter,style=korrektur-done]{überarbeitet}