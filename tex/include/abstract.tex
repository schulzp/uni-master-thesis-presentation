\begin{abstract}
	In this thesis, we explore different modes of physical interaction for sport climbing in \gls{VR}. Therefore, we conducted an experiment to compare the manifestation of stress, anxiety, and presence at subjective and physiological level. The experiment consisted of three conditions: climbing in reality (\gls{top-rope}), and climbing in \gls{VR} (\gls{free-solo}), once with hand and footholds as props and once by using game controllers. All climbing happened in a (simulated) height of \SI{10}{\meter}. We implemented an infrared overlay and utilized tracking to visualize hands and feet in \gls{VR}, respectively. The subjects gave self-reports on anxiety and presence and we measured \gls{HR}, \gls{HRV}, \gls{RR}, and \gls{EDA}. These biosignals were fused and recorded with a custom-made Android App. Participants were 28 average climbers (13 female, age 15-55). At the level of subjective experience, we found that climbing with props in \gls{VR} resulted in significantly higher reports of anxiety than climbing in reality. Regarding the \gls{VR} scenarios, the presence was significantly higher for climbing with props instead of controllers. At the physiological level, we only found a decreased \gls{mRRI} for climbing with props. These results indicate that height-induced anxiety in climbing can be reproduce in \gls{VR} and that props are required to do so.
\end{abstract}
\pagebreak