\section{Discussion}
\label{sec:discussion}

The findings generally confirm the positive effect of props on presence \autocite{Meehan2001} in particular for sport climbing in \gls{VR}. Apparently, the participants reported more anxiety when climbing with props (\ref{cd:B}) compared to controllers (\ref{cd:C}). Also, their self-reports show higher anxiety and realness scores for the condition involving props. Hence, we can reject our null hypothesis.

As expected, both climbing conditions \ref{cd:A} and \ref{cd:B} were equally exerting. That is fully supported by \gls{HR} and partially by \gls{RPE} since the latter shows significantly higher scores for \ref{cd:A} and \ref{cd:B}, compared to \ref{cd:C}. As opposed to \gls{HR}, the \gls{RPE} scores are higher in \ref{cd:B} than in \ref{cd:A}. That difference may be attributed to increased anxiety as observed by \textcite{Hardy2007}. The levels of exertions we demanded from our subjects were lower than those of \textcite{Pijpers2003}. In their experiment they measured \glspl{HR} ranging from \SIrange{133}{176}{bpm} compared to \SIrange{98}{108}{bpm} in our study. 

We assumed anxiety to be a psychosomatic indicator for presence and measured it on multiple scales: \gls{HRV}, \gls{SCR}, and \gls{AT}. The \gls{AT} score (M = 2.43, SD = 2.87) for real climbing (\ref{cd:A}) was consistent with that reported by \textcite[M = 2.7, SD = 1.9]{Pijpers2005}. Further, in alignment with our hypothesis, \gls{AT} was significantly higher for \ref{cd:B} than \ref{cd:C}, although we could not find evidence in the physical measurements. For \gls{HRV}, only \gls{mRRI} was significantly lower in \ref{cd:B} compared to \ref{cd:0} (initial resting), so there is no significant difference between the different climbing conditions. Regarding self-reported presence, only the \gls{IPQ} scale for realness showed a higher score for climbing with props compared to climbing with controllers, although the effect was observed with a low power. \textcite{Pijpers2005} observed longer climbing times in their high-anxiety-condition, whereas our participants took longer for climbing the traverse---in particular on their way in---in both \gls{VR} scenarios compared to climbing in reality.

However, it remains unclear, what exactly caused the increase in stress, anxiety, and presence while climbing in \gls{VR}. Judging from the interviews, there were at least two major triggers: exposition to height and a lack of confidence in the \gls{VR} system, especially when their feet were incorrectly visualized. Interestingly enough, one subject---who faced the offset-issue---compared this experience with climbing on brittle rock faces, which requires extra care and focus.

To check for habitation effects, we created a between-subject-sample from the existing records. This sample had three groups---one for \ref{cd:A}, \ref{cd:B}, and \ref{cd:C}---each containing only those subjects started with \ref{cd:A}, \ref{cd:B}, and \ref{cd:C}, respectively. There were no significant effect of group (condition) on any of the signals measured during the conditions. The validity of this approach should not be exaggerated, though, since the group samples are really small.

Surprisingly, the \glspl{SCR} measures, response-count, and conductivity level, appeared to be utterly unaffected by the conditions. There are several possible reasons for this: First, we only looked at \glsfirst{NSR} and not a specific event like \textcite{Brouwer2014, Insko2001}. Since the majority of our subjects climbs regularly, they may be accustomed to height and according to \textcite[\ppno 16]{Slater2007}, \gls{SCR} \gls{NSR} are unsuited for measuring presence in ordinary situations. Second, the placement of the \gls{EDA} electrodes on the shoulder might have been suboptimal after all, as the shoulder still moves too much while climbing.