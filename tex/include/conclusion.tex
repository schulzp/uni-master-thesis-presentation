\section{Conclusion \& Future Work}

We conducted an experiment to test how much different interaction modes of the same \gls{VR} scene cause stress and anxiety in sport climbers. To the best of our knowledge, no such study has been published at the point of writing. Our results are in alignment with  \textcite{Insko2001, Meehan2001}, who also found risen self-reported and “measured” presence when using props. At the same time, we took “the[ir] pit” experiment one step further, by letting our subjects climb, as a “complex whole-body task” \autocite{Pijpers2005} with and without props. Thereby, we could show that climbing \gls{free-solo} in \gls{VR} results in increased stress and anxiety as well as presence. Hence, we were able to reproduce height-induced anxiety in \gls{VR}, as described by \textcite{Hardy2007,Pijpers2006} for real climbing.

However, the results are ambiguous, since the physiological measures are far less clear than the self-reports. Neither in \gls{HRV} nor in \gls{EDA} we could find strong distinctions between the different modes of climbing. We therefore recommend further research towards physiological measures of anxiety during physical activity. Regarding those signals we did measure, we must doubt that height was the only stressor, since there were also occasional technical issues with incorrectly visualized feet.

Nonetheless, we tested novel approaches for interaction mechanics throughout this experiment---although with mixed results. On the one hand, the heel-mounted trackers helped working around loss-of-tracking-issues, but, on the other hand, they turned out overly sensitive to knocks and faulty calibration. Further, by employing a combination of video-overlay and tracked hands, based on the Leap Motion system, we could create very realistic hands that allowed precise grasping and could compensate for the incorrectly visualized feet.

Being climbers ourselves, we are familiar with fear of falling, yet it turned out hard to operationalize. After all, we only measured stress, from which we cannot infer if our subjects were indeed afraid of getting hurt as a consequence of falling. Therefore, we can merely answer our original research question partially saying: mixed-reality results in higher (self-reported) anxiety and presence, though measured (somatic) symptoms for fear of falling do not explain that. So, before thinking of virtual reality \xcancel{therapy} training, more research remains to be done in order to obtain a thorough insight into the interrelation of stress, anxiety, and presence for climbing in \gls{VR}.