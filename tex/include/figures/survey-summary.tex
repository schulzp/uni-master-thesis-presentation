\newgeometry{top=2.7cm,bottom=7cm,inner=3.5cm,outer=5cm}
\begin{landscape}
	\pagestyle{plain}
	\sisetup{text-rm=\sffamily\tinytableface}
	\tabulinesep=0.5em
	\begin{longtabu} to \textheight {
			@{}
			>{\tiny\sffamily\tinytableface}X[1,l]
			>{\tiny\sffamily\tinytableface}X[1,l]
			>{\tiny\sffamily\tinytableface}X[2,l]
			>{\tiny\sffamily\tinytableface}X[3,l]
			>{\tiny\sffamily\tinytableface}X[3,l]
			>{\tiny\sffamily\tinytableface}X[3,l]
			@{}}
		\taburulecolor{table-line}
		\caption[Summary of studies towards climbing and anxiety]{Summary of studies towards climbing and anxiety \autocite[based on][]{Orth2017}}\label{tab:climbing-summary}\\
		\toprule
		\rowfont[l]{\bfseries\scriptsize} Study
		& Sample
		& Design
		& Task
		& Measure
		& Outcome\\\midrule
		\endfirsthead
		\caption[]{\nameref{tab:climbing-summary} (continued)}\\
		\toprule
		\rowfont[l]{\bfseries\scriptsize} Study
		& Sample
		& Design
		& Task
		& Measure
		& Outcome\\\midrule
		\endhead
		\bottomrule\multicolumn{6}{c}{\sffamily Continued on next page}\\\tabuphantomline
		\endfoot
		\bottomrule
		\endlastfoot
		
		\textcite{Pijpers2003} \newline Exp.~1	
		& N = 13, 5~M, 20-30~yrs, no~experience	
		& \multirow{2}{\linewidth}[0.5\abovetabulinesep]{\vfill
			\tnumerateA{
				\item Route design (height)
				\ronumerate{
					\item mean height of foot holds \SI{0.3}{\meter} from the ground
					\item foot holds \SI{3.7}{\meter} from the ground}}}
		& \multirow{2}{\linewidth}[0.5\abovetabulinesep]{\vfill Climb (indoor, artificial, top-rope, flush vertical, 6 hand- and 5 foot-holds, \SI{7}{\meter} height, \SI{3.5}{\meter} width) nr [note: difficulty assumed as easily achievable; participants practiced on route before testing; each trial required 20 sec continuous climbing]}
		& \tnumerate{
			\item Muscle tension (blood lactate, \gls{EMG}) 
			\item state anxiety (\gls{HR})
			\item state anxiety (\gls{STAI}, anxiety thermometer)}
		& 1, 2, 3 significantly increased when climbing in the high condition \\ \tabucline[sline]{1-2}\tabucline[sline]{5-6}				
		
		Exp.~2
		& N = 17, 11~M, 19–26~yrs, no~experience
		& 
		&
		& Movement (hip trajectory) single camera:
		\tnumerate{
			\item GIE
			\item[] [climb time, \gls{HR} and state anxiety]}
		& 1 and climb time significantly increased when climbing in the high condition \\ \midrule
		
		\textcite{Pijpers2005} \newline Exp.~1
		& N = 8~M, 31.4~yrs ±4.81~SD, no~experience
		& \multirow{2}{\linewidth}[12\abovetabulinesep]{\vfill
			\tnumerateA{
				\item Route design (height)
				\ronumerate{
					\item mean height of foot holds \SI{0.4}{\meter} from the ground
					\item foot holds \SI{5.0}{\meter} from the ground}}}
		& \multirow{2}{\linewidth}[9\abovetabulinesep]{\vfill Climb (indoor, artificial, top-rope, flush vertical, flash, \SI{7}{\meter} height, \SI{3.5}{\meter} width, 7 hand- and 6 foot-holds, mean inter-hold distance = \SI{0.15}{\meter}) as fast and as safely as possible without falling: \newline
			[note: difficulty not given but assumed to be easily achievable; participants practiced on low traverse prior to testing and observed an expert model perform the traverse on video; each trial required 2 traversals]}
		& Movement (discrete actions) multi-camera:
		\tnumerate[series=Exp2]{
			\item number of exploratory movements (number of times a hold is touched without use as support)
			\item number of performatory movements
			\item Use of additional holds (two holds not needed to achieve traversal were set into the route)
			\item[] [climb time, \gls{HR} and anxiety data]}
		& 1 and climb time was significantly higher in the high condition compared to the low condition \\ \tabucline[sline]{1-2}\tabucline[sline]{5-6}
		
		Exp.~2
		& N = 15, 13~M, 20.7~yrs ±2.22~SD, no~experience
		& 
		&
		& \tnumerate[resume*=Exp2]{
			\item contact times and movement times}
		& 4 and climb time was significantly higher in the high condition compared to the low condition \\ \midrule
		
		\textcite{Pijpers2006} \newline Exp.~1
		& N = 12~F, 23.0~yrs ±1.12~SD, no~experience
		& \tnumerateA{\item Route design (height) [with movable hold] 
			\ronumerate{\item holds on average \SI{0.36}{\meter} from the ground (t x 4) 
				\item holds \SI{3.69}{\meter} from the ground (t x 4)}}
		& Rest in climbing position (indoor, artificial, top-rope, flush vertical, \SI{7}{\meter} height, \SI{3.5}{\meter} width, 15 hand- and 15 foot-holds), guess maximum reach height from that position by location of movable hold, and
		perform maximum reach move by holding onto movable hold and adjusting its position
		& Maximum reach measured by tape:
		\tnumerate{
			\item perceived
			\item actual
			\item[] [climb time, \gls{HR} and anxiety data]}
		
		& 1, anxiety, and \gls{HR} significantly increased at height\\ %\tabucline[sline]{1-6}
		
		Exp.~2
		& N = 12, 6~F, 20.8~yrs ±3.57~SD, no~experience
		& \tnumerateA{\item Route design (height) [with 30 instead of 11 holds] 
			\ronumerate{\item holds on average \SI{0.36}{\meter} from the ground (t x 4) 
				\item holds \SI{3.69}{\meter} from the ground (t x 4)}}
		& Climb (indoor, artificial, top-rope, flush vertical, \SI{7}{\meter} height, \SI{3.5}{\meter} width, 15 hand- and 15 foot-holds) as fast and as safely as possible without falling \newline
		[note: difficulty not rated but assumed to be easily achievable; participants practiced on route before testing; each trial required 2 traversals]
		& Movement (discrete actions) single camera;
		\tnumerate{
			\item number of performatory actions (hands and feet) 
			\item number of exploratory actions (hands and feet)
			\item[] [climb time, state anxiety]}
		& 1, 2 and climb time increased significantly when climbing at height compared to close to the ground \\ %\pagebreak %\midrule	
		
		\textcite{Hardy2007} \newline Exp.~1
		& N = 10~M, 22.5~yrs ±3.10~SD, experienced (\gls{cg-uiaa} \RN{5}+--\RN{6}+)
		& \tnumerateA{
			\item Climbing style
			\ronumerate{
				\item \gls{lead}
				\item \gls{top-rope}}
			\item Route grade (difficulty)
			\ronumerate{
				\item route at \gls{lead} limit
				\item route two degrees below \gls{lead} limit}}
		& Climb (rock face, single pitch, place gear independent of \gls{lead}/\gls{top-rope}) two routes B.i and B.ii in style A.i and A.ii
		& \tnumerate{
			\item performance anxiety \gls{RCAI} prior to climbing
			\item effort \gls{HR} + \gls{RPE} and \gls{PME} after climb
			\item performance \gls{CPEI}
			\item[] [climb time to standardize effort]}
		& \titemize{
			\item 1 significantly higher before \glsing{lead} compared to \glsing{top-rope}
			\item 2 significantly higher at limit compared to below limit
			\item 3 significantly higher difference between A.i and A.ii in B.i compared B.ii} \\ \tabucline[sline]{1-4} \tabucline[sline]{6-6}
		
		Exp.~2
		& N = 20~M, 27.25~yrs, ±12.35~SD, experienced (\gls{cg-uiaa} \RN{5}+--\RN{7})
		& \tnumerateA{
			\item Climbing style
			\ronumerate{
				\item \gls{lead}
				\item \gls{top-rope}}
			\item Route grade (difficulty)
			\ronumerate{
				\item route at \gls{lead} limit}}
		& Climb (rock face, single pitch, place gear independent of style) route B.i in style A.i and A.ii
		& 
		& Participants median split in high (HCA) and low (LCA) group on cognitive anxiety
		\titemize{
			\item 1 significantly higher in HCA compared to LCA (consistent across somatic anxiety and activation)
			\item 2 significantly higher for \glsing{lead} compared to \glsing{top-rope} (group has no effect)
			\item no difference in effort between groups}\\ \tabucline[sline]{1-4} \tabucline[sline]{6-6}
		
		Exp.~3
		& N = 24~M, 21.96~yrs, ±1.65~SD, experienced (\gls{cg-uiaa} \RN{5}+--\RN{8})
		& \tnumerateA{
			\item Climbing style
			\ronumerate{
				\item \gls{lead}
				\item \gls{top-rope}}
			\item Route grade (difficulty)
			\ronumerate{
				\item route at \gls{lead} limit
				\item route at \gls{lead} limit}}
		& Climb (rock face, single pitch, place gear independent of style) route B.i in style A.i and A.ii and route B.ii in style A.ii
		&
		& \titemize{
			\item 1 significantly reduced (with regard to activation) before \glsing{lead} compared to \glsing{top-rope}
			\item 2, 3 significantly increased for \glsing{lead} compared to \glsing{top-rope}}\\ %\midrule
		
		\textcite{Nieuwenhuys2008}
		& N = 12, 7~M, 24.4~yrs ±1.98~SD, no~experience
		& \tnumerateA{
			\item Route design (height)
			\ronumerate{
				\item holds \SI{0.44}{\meter} from the ground
				\item holds \SI{4.25}{\meter} from the ground}}
		& Climb (indoor, artificial, top-rope, 26 hand- and foot-holds) self-preferred [note: difficulty level assumed to be easily achievable; participants practiced on the route prior to testing]
		& Visual behavior, movement (gaze-location, discrete actions) eye-tracker, single camera;
		\tnumerate{
			\item fixation (duration, number, average duration, duration per location, duration per type, search rate) \newline
			[note: possible fixation locations included handholds, hands, wall, other and possible fixation types were exploratory or performatory]
			\item mean distance of fixation 
			\item movement time (climb time, stationary time, moving time (hands and feet), average movement duration between holds) 
			\item mean distance of hand movements [nb: additional measures of interest were \gls{HR} and anxiety]}
		& \titemize{
			\item Climb time, movement time between holds and time spent static was significantly longer and number of movements were significantly greater in the high condition compared to the low condition;
			\item Fixation durations were significantly longer, number of fixations significantly increased, and search rate significantly decreased in the high condition compared to the low condition.}\\ \midrule%\pagebreak
		
		\textcite{Werts2010}
		& N = 10, 5~M, 34.5~yrs ±5.48~SD, experienced (\gls{cg-uiaa} \RN{4}+--\RN{5})
		& \tnumerateA{
			\item Cause of exertion/anxiety
			\ronumerate{
				\item climbing
				\item hiking}
			\item Point of time
			\ronumerate{
				\item before training
				\item after training}}
		& Climb (rock face, top-rope) and jump (repeatedly) with self-preferred amount of slack;\newline
		Baseline: hiking
		& Anxiety, performance single camera;
		\tnumerate{
			\item \gls{HR}, and salivary cortisol level (pre, throughout, and post training)
			\item memorization/problem solving capacity
			\item performance
			\item[] [state anxiety (\gls{STAI} pre, throughout, and post training, anxiety thermometer)]}
		& \titemize{
			\item for women: state anxiety (\gls{STAI}), worry, and somatic anxiety (\gls{WAI-T}) significantly lower after training
			\item 1 \gls{HR} significantly higher for B.i compared to B.ii, always higher (difference) in A.i than A.ii
			\item 2 no significant effects
			\item 3 no significant effects across ratings}\\ \midrule			
		
		\textcite{Bertle2014}
		& N = 42, 23~F, experienced
		& \tnumerateA{
			\item Height above ground
			\ronumerate{
				\item warm up route \SIrange{8}{10}{\meter}
				\item main route \SIrange{20}{30}{\meter}}
			\item Climbing style [skill level \textgreater French 7b+]
			\ronumerate{
				\item \gls{top-rope} (= low fall height) [or \glsing{lead}]
				\item \gls{lead} (= high fall height) [or \glsing{lead} skipping every other bolt]}}
		& Climb (outdoor, artificial) A.i then A.ii in style B.i/ii (randomized),  \newline
		[note: routes were picked by the participants, so they estimated their chance in succeeding with \SI{75}{\percent}]
		& Appraisal dimensions (\gls{GAF} questionnaire) before/after climb judging before/after jump in retrospective, movement (activity), behavior (specific movements) single camera;
		\tnumerate{
			\item psychological exertion
			\item behavior (gaze, hesitation before jump, checking equipment)
			\item \gls{HR} and \gls{AHR}
			\item[] [\gls{EDA} on angle, climb time (less \SIrange{10}{20}{\percent} for placing gear), activity (integral over movements in any direction \textgreater \SI{2}{\second})]}
		& \titemize{
			\item A and B account for same effects in 1, 2 except for \gls{HR}, \gls{AHR}, and the feeling of overexertion;
			\item 1 significantly higher (except for overexertion) in A.i compared to A.ii
			\item 2 significantly longer hesitation in B.ii compared to B.i
			\item 3 significantly higher in A.ii, and B.ii compared to A.i, and B.i}
	\end{longtabu}
	\aftergroup\restoregeometry
\end{landscape}