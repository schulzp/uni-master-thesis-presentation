\section{Study}
\label{sec:study}

Based on the experiment concepts introduced earlier, we designed a study to test our hypothesis, that using props increases anxiety of sport climbers when climbing in \gls{VR} and, therefore, their sense of presence, see Section \ref{sec:hypothesis}. To examine this effect, we created two different modes of climbing in \gls{VR}, one with props and one without, see Section \ref{sec:impl}.

% p o. r - pearsonscher Korrelationskoeffizient [-1.0: Effekt ~ 1/Auslöser; +1.0: Effekt ~ Auslöser]
% d - effect stregth [0.3-0.5: small; 0.5-0.8: medium; > 0.8: strong effect]
% tails - 1: directed hypothesis; 2: non-directed hypothesis (split confidence/alpha niveau by two)
% type-1-error $\alpha$ - p(reject $H_0$|$H_0$ is true) for example $H_0$: result = avg; $\alpha$ to low $\rightarrow$ chance of type-2-error increases
% type-2-error $\beta$ - p(retain $H_0$|$H_0$ is false) $\rightarrow$ increase sample size
% treatment - same as condition in context of an repeated measures ANOVA

Since anxiety is highly subjective and varies considerably between participants, a within-subject experimental study was chosen to test our hypothesis, in alignment with the designs of earlier, similar studies summarized in \namedvref{sec:related-work}.

\subsection{Variables and Conditions}
\label{sec:variables-conditions}

Our study design stipulates two independent variables: \textbf{\sffamily \namedlabel{iv:props}{I\textsubscript{p}}} props $\in$ \{with; without\}, and
\textbf{\sffamily \namedlabel{iv:vision}{I\textsubscript{v}}} vision $\in$ \{real; virtual\}. With three conditions, see Table \vref{tab:conditions}, we wanted to examine their effect on the following dependent variables: \textbf{\sffamily \namedlabel{dv:exertion}{D\textsubscript{e}}} physical exertion, \textbf{\sffamily \namedlabel{dv:anxiety}{D\textsubscript{a}}} anxiety, and
\textbf{\sffamily \namedlabel{dv:presence}{D\textsubscript{p}}} presence. The order of the conditions was randomized using a Williams Design \autocites{LatinSquaresWilliams}[based on][]{Williams1949}, so we ended up with six different permutations.

\begin{table}[htb]
	\caption[Overview of conditions with related variables]{Conditions with independent variables vision (\ref{iv:vision}), and props (\ref{iv:props}), as well as expected outcomes for dependent variables physical exertion (\ref{dv:exertion}), anxiety (\ref{dv:anxiety}), and presence (\ref{dv:presence})}
	\label{tab:conditions}
	\tablestyle{
	\begin{tabu} to \columnwidth {
			@{}
			X[-1]
			X[1]
			X[-1]
			X[-1]
			X[-1]
			X[-1]
			X[-1]
			@{}
		}
		\toprule
		           \multicolumn{2}{c}{\textbf{Condition}}            & \multicolumn{2}{c}{\textbf{Independent Var.}} &        \multicolumn{3}{c}{\textbf{Dependent Var.}}        \\
		\textbf{\#}                   & \textbf{Movement}                             & \textbf{\ref{iv:vision}} & \textbf{\ref{iv:props}}             & \textbf{\ref{dv:exertion}} & \textbf{\ref{dv:anxiety}} & \textbf{\ref{dv:presence}} \\ \midrule
		\namedlabel{cd:0}{C\textsubscript{0}} & Initial resting phase                & real            & -                          & low               & low              & -                 \\ \tabucline[sline]{1-7}
		\namedlabel{cd:A}{C\textsubscript{A}} & Climbing at real altitude            & real            & with                       & high              & high             & high              \\
		\namedlabel{cd:B}{C\textsubscript{B}} & Climbing at virtual altitude         & virtual         & with                       & high              & medium           & high              \\
		\namedlabel{cd:C}{C\textsubscript{C}} & By controllers at virtual altitude & virtual         & without                    & low               & low              & medium            \\ \bottomrule
	\end{tabu}}
\end{table}


\subsection{Measurements} 
\label{sec:study-measurements}

The sense of presence in \gls{VR} is normally measured with questionnaires, behavioral analysis, or physiological measures \autocite[\ppno 13]{Slater2007}. We went with the first two options an in alignment with \textcite{Slater2007,Slater2004a}, and regarded stress and anxiety as indicators of presence for this experiment.

\begin{table}[htb]
	\caption[Overview of measurements and their related dependent variables]{Our measurements and their related dependent variables: physical exertion (\ref{dv:exertion}), anxiety (\ref{dv:anxiety}), and presence (\ref{dv:presence}); $\bullet$ and $\circ$ indicate direct, and indirect measures}
	\label{tab:measuremets}
	\tablestyle{\begin{tabu} to \textwidth {
				@{}
				X[-1]
				X[-1]
				X[-1]
				X[-1]
				X[-1]
				@{}
			}
		\toprule
		            \multicolumn{2}{c}{\textbf{Measurement}}             &        \multicolumn{3}{c}{\textbf{Dependent Var.}}        \\
		\textbf{Name}                & \textbf{When}                                      & \textbf{\ref{dv:exertion}} &  \textbf{\ref{dv:anxiety}} & \textbf{\ref{dv:presence}} \\ \midrule
		duration                & for \ref{cd:A}, \ref{cd:B}, \ref{cd:C}    &                   &                  &                   \\
		\glsfirst{HR}       & during \ref{cd:A}, \ref{cd:B}, \ref{cd:C} & $\bullet$         &  $\circ$         & $\circ$           \\
		\glsfirst{RR}       & during \ref{cd:A}, \ref{cd:B}, \ref{cd:C} & $\bullet$         &  $\circ$         & $\circ$           \\
		\glsfirst{HRV}      & during \ref{cd:A}, \ref{cd:B}, \ref{cd:C} &                   &  $\bullet$       & $\circ$           \\
		\glsfirst{EDA}      & during \ref{cd:A}, \ref{cd:B}, \ref{cd:C} &                   &  $\bullet$       & $\circ$           \\
		\glsfirst{AT}       & after \ref{cd:A}, \ref{cd:B}, \ref{cd:C}  &                   &  $\bullet$       &                   \\
		\glsfirst{RPE}      & after \ref{cd:A}, \ref{cd:B}, \ref{cd:C}  & $\bullet$         &                  &                   \\
		\glsfirst{IPQ}      & after \ref{cd:B}, \ref{cd:C}              &                   &                  & $\bullet$         \\
		\glsfirst{STAI}     & upfront                                   &                   &  $\bullet$       &                   \\
		\glsfirst{vHI}      & upfront                                   &                   &  $\bullet$       &                   \\ \bottomrule
	\end{tabu}}
\end{table}


Our primary stress and anxiety measures were \glsfirst{HRV} \autocite{Castaldo2015, Hynynen2009} and \glsfirst{SCR} \autocite{Brouwer2014,Meehan2001}. Additionally, presence was measured using the widespread \glsfirst{IPQ} with the following scales 
\begin{inlinelist}
	\item \textit{spatial presence (sp),} the sense of being physically present in the VE
	\item \textit{involvement (inv),} the attention devoted to the VE and the involvement experienced
	\item \textit{experienced realism (real),} the subjective experience of realism in the VE 
\end{inlinelist} \autocite{IPQ2016,VanBaren2004}.

\textcite{Meehan2001} also suggest using \gls{HR}, \gls{RR}, and \gls{EDA} as indicators for presence. Regarding sport climbers, \textcite{Hardy2007} reckon that those signals---when elevated---may be interpreted as signs of somatic anxiety, too. However, \gls{HR} and \gls{RR} cannot be described as unitary functions of arousal, because they are also functions of movement \autocites{Croft2004}[based on][]{Obrist1981}. Hence, they can only be interpreted as indicators for somatic anxiety when comparing conditions with a similar level of movement. In our setup these are only \ref{cd:A} and \ref{cd:B}. As a control variable for the level of movement, we also asked the subjects to give a \glsfirst{RPE}, on a labeled \glsfirst{VAS} from 6 to 20, developed by \textcite{Borg2004}.

 For measuring the severity of acrophobia of our subjects, we employed the \glsfirst{vHI} questionnaire by \textcite{Huppert2017}. To meter the a priori trait anxiety of subjects we used the \glsfirst{STAI} by \textcite{Grimm2009}. After each condition, we also tested the manifestation of anxiety at the level of subjective experience through a \gls{VAS} called \glsfirst{AT} \autocites{Houtman1989}[also called fear thermometer by][]{Antony2005}. This is a quick and validated alternative to more comprehensive tools such as \gls{SUDS}, or \gls{CSAI-2} \autocite{Pijpers2006}. It is a continuous scale ranging from 0 (not anxious at all) to 10 (extremely anxious) displayed as a box of \SI{10}{\centi\meter} length, in which participants place a mark to indicate their level of anxiety \autocite{Houtman1989}.

\subsection{Procedure and Tasks}

Our experiment procedure included the following steps
\begin{inlinelist}
	\item briefing
	\item put on climbing harness 
	\item fit with \gls{HR}, \gls{RR}, and \gls{EDA} electrodes
	\item initial rest (\ref{cd:0}) and upfront questionnaires
	\item conditions \ref{cd:A}, \ref{cd:B}, and \ref{cd:C}
	\item debriefing
\end{inlinelist}. After each condition the participants had time to recuperate while filling in the questionnaires, see also Table \vref{tab:measuremets}.

The primary task for all conditions was traversing a climbing wall, from a start point to a return point and back. The subjects were only belayed in condition \ref{cd:A} utilizing an auto-belay-device. To ensure that the participants are aware of their exposition to height, we introduced secondary tasks:
\begin{inlinelist}
	\item \textit{before climbing,} throw a ball down to the ground and read out the number displayed next to the impact site
	\item \textit{while resting at the return point,} look down and read out numbers shown near the ground floor.
\end{inlinelist}

\subsection{Sample Composition}

Selecting an appropriate sample size is a crucial step in designing a successful study \autocite{Guo2013}. Therefore, we used the tool \href{http://glimmpse.samplesizeshop.org/}{GLIMMPSE}, developed by \citeauthor{Guo2013}, to calculate an appropriate sample size. The output listed in Table \vref{tab:glimmpse} is based on the outcome of similar, climbing-related experiments by \textcite{Hardy2007}. In this particular study, they analyzed seven dependent variables with four repeated measures. To reproduce these results with a maximum type \RN{1} error of α = 0.05 and a power > 0.8, a minimum sample of 11 is required. In contrast, a worst case scenario---with less different mean values and higher variance---would require a minimum sample of 28.

\begin{table}[htb]
	\centering
	\captionsetup{subrefformat=parens}
	\caption[Sample size calculation results]{Sample size calculation results of two independent tools, \href{http://glimmpse.samplesizeshop.org/}{GLIMMPSE} \subref{tab:glimmpse} and  \href{http://www.gpower.hhu.de/}{G*Power} \subref{tab:gpower}, using reference values from \textcite{Hardy2007,Pijpers2006}} 
	\label{tab:sample-size}
	\begin{subtable}[t]{0.49\textwidth}
	\caption{GLIMMPSE sample size results}
	\centering
	\tablestyle{
		\label{tab:glimmpse}
		\begin{tabular}{@{}
		l
		l
		S[table-format=1.1]
		S[table-format=1.3]
		S[table-format=1.0]@{}}
			\toprule
			{\thead{α}}                   & {\thead{f\textsubscript{v}}} & {\thead{f\textsubscript{m}}} & {\thead{Power}} & {\thead{Sample}} \\ \midrule
			\multirow{9}{*}[-0.5em]{0.05} & \multirow{3}{*}{0.5}                  & 0.8                            & 0.815           & 9                     \\
			                              &                                       & 1.0                            & 0.852           & 7                     \\
			                              &                                       & 1.2                            & 0.899           & 6                     \\
			\cmidrule{2-5}                & \multirow{3}{*}{1.0}                  & 0.8                            & 0.831           & 16                    \\
			                              &                                       & 1.0                            & 0.824           & 11                    \\
			                              &                                       & 1.2                            & 0.867           & 9                     \\
			\cmidrule{2-5}                & \multirow{3}{*}{2.0}                  & 0.8                            & 0.805           & 28                    \\
			                              &                                       & 1.0                            & 0.810           & 19                    \\
			                              &                                       & 1.2                            & 0.812           & 14                    \\ \bottomrule
             \multicolumn{5}{l}{\parbox{0.9\textwidth}{\vspace{0.2cm}\footnotesize \textcolor{gray}{f\textsubscript{m} and f\textsubscript{v} are scale factors for mean values and variance to simulate different scenarios}}}
		\end{tabular}
}
\end{subtable}

	\begin{subtable}[t]{0.49\textwidth}
	\caption{G*Power a priori sample size results}
	\label{tab:gpower}
	\robustify\bfseries
	\centering
	\tablestyle{
	\begin{tabular}{@{}lS[table-format=2.2]S[table-format=3.2]@{}}
		\toprule
		\multirow{2}{*}[-0.5em]{\thead{Output}} & \multicolumn{2}{c}{\thead{Nonsphericity corr. ε}} \\
		\cmidrule{2-3}                          & \bfseries 1 & \bfseries 0.75                      \\ \midrule
		Noncentrality λ               & 10.78       & 9.55                               \\
		Critical F                              & 3.21        & 3.57                                \\
		Numerator df                            & 2           & 1.5                                \\
		Denominator df                          & 42          & 37                                  \\
		Actual power                            & 0.81        & 0.80                                \\
		Total sample size                       & 22         & 26                                  \\ \bottomrule
	\end{tabular}}
\end{subtable}

\end{table}

Besides GLIMMPSE, \href{http://www.gpower.hhu.de/}{G*Power 3.1} was used to calculate an appropriate sample size for testing single pairs using univartiate \gls{ANOVA}. With α = 0.05, power > 0.8, and an assumed effect size \autocite[cf.][]{Cohen1988} d > 1.2 (based on the results of the GLIMMPSE analysis), the sample size ranges from 11 to 14, for nonsphericity correction ε between 1.0 and 0.75 (worst case, compare \textcite{PowerGuide}). Table \ref{tab:gpower} shows even worse scenarios, with a weaker effect of d = 0.7. Those would require 24 to 26 participants, and, therefore, we wanted to get 24 to 30 participants, to run each of the six permutations of condition order (see also \namedvref{sec:variables-conditions}) equally frequent.

To forestall effects of skill-imposed self-confidence \autocite{Hardy2007}, we wanted a sample with climbers of average skill level, which correlates to a range from 5+ to 7+, according to the manager of the local climbing gym where the experiment was hold.